% compile with XeLaTeX or LuaLaTeX
\input{preamble}
\setdefaultlanguage{french}

\title{Problèmes pour enfants de 5 à 15 ans}

\author{V.\,I.~Arnold
\vspace*{2cm}\\
\includegraphics[width=\linewidth]{resources/photo-arnold_small}
}
\date{}

\begin{document}
\maketitle
\thispagestyle{empty}
\cleardoublepage
\setcounter{page}{1}
\begin{abstract}
Ce document est constitué de 77 problèmes, rédigés ou sélectionnés par l'auteur, pour aider au développement d'une culture de la réflexion.
La plupart de ces problèmes ne requiert pas de connaissance spécifique.
Pour autant, la résolution de certains de ces problèmes peut s'avérer difficile même pour des professeurs.

Ce livre est destiné aux élèves, étudiants, enseignants, parents -- à tous ceux qui considèrent la culture de la réflexion comme une part essentielle du développement de la personnalité.
\end{abstract}
\clearpage

\section*{Préface}
J'ai couché ces problèmes sur le papier au printemps 2004 à Paris, lorsque des Parisiens russes m'ont demandé d'aider
leurs jeunes enfants à acquérir la culture de la pensée traditionnelle russe.

Je suis profondément convaincu que cette culture se cultive avant tout grâce à une réflexion indépendante et précoce.
sur des questions simples, mais pas faciles, semblables à celles ci-dessous (les problèmes 1, 3, 13 sont les plus recommandés).

Ma longue expérience a montré que, très souvent, les imbéciles qui prennent du retard à l'école les résolvent
mieux que les élèves de première, car - pour leur survie au fond de la classe - ils doivent en permanence
penser plus que ce qui est nécessaire \enquote{pour gouverner toute Séville et Grenade}, comme Figaro avait l'habitude de dire de lui-même, alors que les élèves de première ne peuvent pas saisir \enquote{ce qui doit être multiplié par quoi} dans ces problèmes.

J'ai aussi remarqué que les enfants de cinq ans résolvent mieux des problèmes similaires que les élèves gâtés
par le coaching, qui à leur tour gèrent mieux les questions que les étudiants
habitués à la besogne universitaire qui de toute façon battent leurs professeurs (les pires pour résoudre ces problèmes simples sont
les lauréats du prix Nobel et du prix Fields).
\clearpage
\section*{Les problèmes}

\begin{problem}{1.}
	Il a manqué à Masha sept kopecks pour s'acheter son premier livre de lecture.
	À Misha, il ne manquait qu'un kopeck.
	Ils ont alors décidé de mettre leur argent en commun pour cet achat, mais même là ils
	n'avaient pas assez pour acheter le livre.
	Combien coûtait le livre ?
\end{problem}

\begin{problem}{2.}
	Une bouteille avec un bouchon coûte 10 kopecks, alors que la bouteille coûte 9 kopecks de
	plus que le bouchon.
	Combien coûte la bouteille sans le bouchon ?
\end{problem}

\begin{problem}{3.}
	Une brique pèse une livre et la moitié d'une brique. Combien de livres pèse une brique ?
\end{problem}

\begin{problem}{4.}
	Une cuiller de vin est transvasée d'un tonneau de vin dans un verre de thé (non plein).
	Ensuite, on transvase une cuiller (la même) du mélange (inhomogène) du verre vers le tonneau.
	À ce moment, il y a dans le tonneau comme dans le verre une certaine quantité du liquide étranger
	(du vin dans le verre et du thé dans le tonneau).
	Dans quel récipient le volume du liquide étranger est-il le plus grand : dans le verre ou dans
	le tonneau ?
\end{problem}

\begin{problem}{5.}
	Deux femmes âgées partent du point $A$ au point $B$ (et respectivement du point $B$ au point $A$)
	à l'aube (cheminant sur la même route).
	Elles se croisent à midi, ne s'arrêtent pas et continuent leur chemin à une vitesse constante.
	La première femme arrive au point $B$ à 16h et la seconde rejoint le point $A$ à 21h.
	À quelle heure le soleil s'est-il levé ce jour-là ?
\end{problem}

\begin{problem}{6.}
	L'hypoténuse d'un triangle rectangle (dans un examen américain standard) est de 10~pouces,
	la hauteur depuis l'hypoténuse est de 6~pouces. Trouvez l'aire du triangle.

	Des élèves américains ont réussi à résoudre ce problème pendant plus d'une décennie.
	Mais ensuite, des élèves russes sont arrivés de Moscou, et aucun d'entre eux n'a été capable de le résoudre comme 		l'avaient fait leurs pairs américains
	(en donnant 30~pouces carrés comme réponse). Pourquoi ?
\end{problem}

\begin{problem}{7.}
	Vasya a deux s\oe urs de plus qu'il n'a de frères.
	Combien les parents de Vasya ont-ils de filles de plus que de fils ?
\end{problem}

\begin{problem}{8.}
	En Amérique du Sud se trouve un lac circulaire.
	Chaque année, le 1er Juin, une fleur nommée Victoria Regia
	apparaît en son centre (sa tige s'élève des profondeurs et ses pétales se posent à la surface
	comme pour ceux d'un nénuphar).
	Chaque jour, la surface couverte par la fleur double et le 1er Juillet la fleur recouvre
	entièrement le lac, perds ses pétales, et ses graines tombent au fond de l'eau.
	À quelle date la fleur recouvrait-elle la moitié de la surface du lac ?
\end{problem}

\begin{problem}{9.}
	Un paysan doit faire traverser une rivière à un loup, une chèvre et un chou.
	Malheureusement, son bâteau est si petit qu'il ne peut emporter qu'un passager avec lui à
	chaque traversée.
	Comment peut-il transporter tout le monde de l'autre côté de la rivière ?
	(Le loup ne peut pas être laissé seul avec la chèvre et la chèvre avec le chou pas davantage).
\end{problem}

\begin{problem}{10.}
	En une journée, un escargot grimpe de \SI{3}{\cm} sur une planche.
	Pendant la nuit, il s'endort et redescend de \SI{2}{\cm}.
	La planche mesure \SI{10}{\metre} de hauteur, et une friandise attend l'escargot à son sommet.
	Combien de jours mettra l'escargot à atteindre la friandise ?
\end{problem}

\begin{problem}{11.}
	Un garde chasse a quitté sa tente pour marcher \SI{10}{\km} vers le sud, a tourné vers l'est,
	a marché \SI{10}{\km} vers l'est,
	a rencontré son ami l'ours, a tourné vers le nord et après \SI{10}{\km} de marche est arrivé à sa tente.
	Quelle était la couleur de l'ours et où cela s'est-il passé ?
\end{problem}

\begin{problem}{12.}
	La marée était haute aujourd'hui à midi. À quelle heure sera-t-elle haute (au même endroit) demain ?
\end{problem}

\begin{problem}{13.}
	Deux livres de Pushkin, le premier et le deuxième, sont côte à côte sur une étagère.
	Les pages de chaque volume ont une épaisseur de \SI{2}{\cm} pour chaque volume,
	et l'épaisseur de la couverture est de \SI{2}{\mm} chacune pour la page de garde et la quatrième
	de couverture.
	Un ver a grignoté de la première page du premier livre à la dernière du deuxième livre, en
	se déplaçant perpendiculairement aux pages.
	Quelle est la longueur du parcours du vers ?
	[Ce problème de topologie dont la réponse est, de manière surprenante, \SI{4}{\mm}, semble
	absolument impossible à des esprits académiques, mais certains élèves de maternelle
	s'en sortent sans souci.]
\end{problem}

\begin{problem}{14.}
	Trouve une forme dont les vues de dessus et de face correspondent aux figures ci-dessous.
	Dessine cette forme en vue de côté (en faisant apparaître les arêtes cachées en pointillés).
	\begin{figure}
		\footnotesize
		\null\hfill
		\parbox{0.2\linewidth}{\centering\includegraphics{resources/taskbook-99}\\Vue de dessus}
		\hfill
		\parbox{0.2\linewidth}{\centering\includegraphics{resources/taskbook-98}\\Vue de face}
		\hfill\null
	\end{figure}
\end{problem}

\begin{problem}{15.}
	Combien y a-t-il de façons de décomposer le nombre 64 en une somme de dix entiers positifs dont
	le maximum est 12 ?
	[Des décompositions qui ne diffèrent que par l'ordre des entiers à sommer ne sont pas considérés
	comme différents.]
\end{problem}

\begin{problem}{16.}
	En empilant quelques pavés droits l'un sur l'autre (par exemple, des dominos),
	on peut faire dépasser une longueur $x$.
	Quelle est la longueur maximale que l'on peut laisser dépasser sans que l'édifice ne s'effondre ?
	\begin{figure}
		\includegraphics{resources/taskbook-97}
	\end{figure}
\end{problem}

\begin{problem}{17.}
	Les villes $A$ et $B$ son séparées de \SI{40}{\km}.
	Deux cyclistes quittent respectivement les villes $A$ et $B$ au même moment,
	en se dirigeant l'un vers l'autre, l'un à une vitesse de
	\SI{10}{\km\per\hour} et l'autre à la vitesse de \SI{15}{\km\per\hour}.
	Une mouche s'élance du point $A$ en même temps que le premier cycliste à une
	vitesse de \SI{100}{\km\per\hour}, rencontre le second cycliste,
	touche son casque et fait demi-tour vers le premier, jusqu'à toucher son
	casque et faire demi-tour, ainsi de suite jusqu'à ce que les casques des
	cyclistes se percutent et écrasent la mouche.
	Au total, combien de kilomètres a parcouru la mouche ?
	\begin{figure}
		\includegraphics{resources/taskbook-1}
	\end{figure}
\end{problem}

\begin{problem}{18.}
	Un domino couvre deux cases d'un échiquier.
	Recouvre toutes les cases de l'échiquier sauf les les deux opposées
	(situées sur la même diagonale) avec 31 dominos.
	[Un échiquier est constitué de $8 \times 8 = 64$ cases.]
	\begin{figure}
		\includegraphics{resources/taskbook-2}
	\end{figure}
\end{problem}

\begin{problem}{19.}
	Une chenille souhaite ramper d'un coin (au sol à gauche) d'une chambre
	cubique au coin opposé (au plafond à droite).
	Quel est le plus court chemin pour un tel trajet le long des murs de
	la chambre ?
	\begin{figure}
		\includegraphics{resources/taskbook-3}
	\end{figure}
\end{problem}

\begin{problem}{20.}
	Tu as deux récipients de 5~litres et 3~litres.
	Mesure un volume d'un litre dans l'un des récipients sans utiliser d'autre
	outil de mesure.
	\begin{figure}
		\includegraphics{resources/taskbook-4}
	\end{figure}
\end{problem}

\begin{problem}{21.}
	Il y a 5 têtes et 14 jambes (ou pattes) dans une famille.
	Combien y a-t-il de personnes et de chiens dans cette famille ?
\end{problem}

\begin{problem}{22.}
	Des triangles équilatéraux reposent sur les côtés $AB$, $BC$ et $CA$
	d'un triangle $ABC$.
	Montre que leurs centres ($*$) forment un triangle équilatéral.
	\begin{figure}
		\includegraphics{resources/taskbook-6}
	\end{figure}
\end{problem}

\begin{problem}{23.}
	Quels polygones peut-on obtenir à partir de l'intersection d'un cube
	avec un plan ?
	Peut-on obtenir un pentagone ? Un heptagone ? Un hexagone régulier ?
	\begin{figure}
		\includegraphics{resources/taskbook-7}
	\end{figure}
\end{problem}

\begin{problem}{24.}
	Tirez une ligne droite passant par le centre d'un cube de telle sorte que la
	somme des carrés des distances entre cette ligne et les huit sommets du cube
	soit a) maximale, b) minimale (par rapport à d'autres lignes similaires).
\end{problem}

\begin{problem}{25.}
	Un cône circulaire droit est coupé par un plan selon une courbe fermée. Deux
	sphères inscrites dans le cône sont tangentes au plan, l'une au point $A$ et
	l'autre au point $B$. Trouvez un point $C$ sur la ligne de coupe de telle
	sorte que la somme des distances $CA + CB$ soit a) maximale, b) minimale.
	\begin{figure}
		\includegraphics{resources/taskbook-9}
	\end{figure}
\end{problem}

\begin{problem}{26.}
	La surface de la Terre est projetée sur un cylindre formé par les lignes
	tangentes aux méridiens en leurs points équatoriaux, selon les rayons
	parallèles à l'équateur passant par l'axe des pôles de la Terre.
	L'aire de la projection de la France sera-t-elle plus grande ou plus petite
	que l'aire réelle de la France ?
	\begin{figure}
		\includegraphics{resources/taskbook-10}
	\end{figure}
\end{problem}

\begin{problem}{27.}
	Démontrez que le reste de la division du nombre $2^{p-1}$ par un nombre
	premier impair $p$ est $1$.
	(Exemples : $2^2 = 3a + 1$, $2^4 = 5b+1$, $2^6 = 7c+1$, $2^{10} - 1 = 1023 =
	11\cdot 93$.)
\end{problem}

\begin{problem}{28.}
	Une aiguille de \SI{10}{\cm} est jetée aléatoirement sur une feuille de
	papier lignée où la distance entre les lignes adjacentes est également de
	\SI{10}{\cm}. Cela est répété $N$ fois (un million).
	Combien de fois (approximativement, avec une erreur de quelques pourcents)
	l'aiguille traversera-t-elle une ligne du papier ?
	\begin{figure}
		\includegraphics{resources/taskbook-12}
	\end{figure}
	On peut réaliser (comme je l'ai fait à l'âge de 10 ans) cette expérience
	avec $N=100$ au lieu d'un million de lancers.
	[La réponse à ce problème est surprenante : $\frac2{\pi}N$. De plus, même
	pour une aiguille courbée de longueur $a \cdot \SI{10}{\cm}$, le nombre
	d'intersections observé sur $N$ lancers sera approximativement
	$\frac{2a}{\pi}N$.
	Le nombre $\pi \approx  \frac{355}{113} \approx \frac{22}7.$]
\end{problem}

\begin{problem}{29.}
	Les polyèdres ayant des faces triangulaires sont, par exemple, les solides
	platoniciens : tétraèdre (4 faces), octaèdre (8 faces), icosaèdre (20 faces
	– et toutes les faces sont identiques ; il est intéressant de le dessiner :
	il a 12 sommets et 30 arêtes).
	\begin{figure}
		\footnotesize
		\null\hfill
		\parbox{0.3\linewidth}{\centering\includegraphics{resources/taskbook-131}\\Tétraèdre
		($\text{tétra}= 4$)}
		\hfill
		\parbox{0.3\linewidth}{\centering\includegraphics{resources/taskbook-132}\\Octaèdre
		($\text{octo}= 8$)} \hfill\null\\
		{\Huge ?}\\Icosaèdre
	\end{figure}
	Est-il vrai que pour tout polyèdre convexe borné ayant des faces
	triangulaires, le nombre de faces est égal à deux fois le nombre de sommets
	moins quatre ?

	Un autre solide platonicien (il y en a cinq en tout) :
	\begin{figure}
		\includegraphics{resources/taskbook-14}
	\end{figure}
\end{problem}

\begin{problem}{30.}
	Un dodécaèdre est un polyèdre convexe ayant douze faces pentagonales
	régulières, vingt sommets et trente arêtes (ses sommets sont les centres des
	faces d'un icosaèdre).
	Inscrivez dans un dodécaèdre cinq cubes (les sommets de chaque cube étant
	des sommets du dodécaèdre) dont les arêtes sont les diagonales des faces du
	dodécaèdre (un cube a 12 arêtes, une par face).
	[Cette construction a été inventée par Kepler pour représenter les
	planètes.]
\end{problem}

\begin{problem}{31.}
	Trouvez l'intersection de deux tétraèdres inscrits dans un cube (de sorte
	que les sommets de chaque tétraèdre soient des sommets du cube et que leurs
	arêtes soient les diagonales des faces).
	Quelle fraction du volume du cube est contenue dans l'intersection des
	tétraèdres ?
\end{problem}

\begin{problem}{31\textsuperscript{bis}.} Construisez la section d'un cube par
	un plan passant par trois points donnés sur les arêtes.
	[Tracez le polygone correspondant à l'intersection plane avec les faces du
	cube.]
	\begin{figure}
		\includegraphics{resources/taskbook-15}
	\end{figure}
\end{problem}

\begin{problem}{32.}
	Combien de symétries un tétraèdre possède-t-il ? Et un cube ? un octaèdre ?
	un icosaèdre ? un dodécaèdre ? Une symétrie est une transformation
	conservant les longueurs.
	Parmi elles, combien sont des rotations et combien sont des réflexions (pour
	chacun des cinq cas listés) ?
\end{problem}

\begin{problem}{33.}
	Combien de façons existe-t-il pour peindre les $6$ faces d'un cube similaire
	avec six couleurs $(1,\dotsc,6)$ [une couleur par face] de telle sorte
	qu'aucun des cubes colorés obtenus ne soit identique (c'est-à-dire qu'il ne
	puisse être transformé en un autre par une rotation) ?
	\begin{figure}
		\includegraphics{resources/taskbook-17}
	\end{figure}
\end{problem}

\begin{problem}{34.}
	Combien existe-t-il de façons différentes de permuter $n$ objets ? Il y en a
	six pour $n=3$ : $(1,2,3)$, $(1,3,2)$, $(2,1,3)$, $(2,3,1)$, $(3,1,2)$,
	$(3,2,1)$. Combien y en a-t-il si le nombre d'objets est $n=4$ ? $n=5$ ?
	$n=6$ ? $n=10$ ?  
	\begin{figure}
		\includegraphics{resources/taskbook-18}
	\end{figure}
\end{problem}

\begin{problem}{35.}
	Un cube possède $4$ diagonales principales. Combien de permutations
	différentes de ces quatre objets peuvent être obtenues par rotations du cube
	?
	\begin{figure}
		\includegraphics{resources/taskbook-19}
	\end{figure}
\end{problem}

\begin{problem}{36.}
	La somme des cubes de trois entiers est soustraite du cube de la somme de
	ces nombres. La différence est-elle toujours divisible par $3$ ?
\end{problem}

\begin{problem}{37.}
	Même question pour les cinquièmes puissances et la divisibilité par $5$, et
	pour les septièmes puissances et la divisibilité par $7$.
\end{problem}

\begin{problem}{38.}
	Calculez la somme suivante :
	\begin{equation*}
		\frac{1}{1\cdot 2} + \frac{1}{2\cdot 3} + \frac{1}{3\cdot 4} + \dotsb + \frac{1}{99\cdot 100}
	\end{equation*}
	(avec une erreur ne dépassant pas $1\%$ de la réponse).
\end{problem}

\begin{problem}{39.}
	Si deux polygones ont des aires égales, alors ils peuvent être découpés en
	un nombre fini de parties polygonales, qui peuvent ensuite être réarrangées
	pour obtenir à la fois le premier et le second polygone. Prouvez cela !
	[Pour les solides dans l’espace, ce n’est pas le cas : un cube et un
	tétraèdre de volumes égaux ne peuvent pas être découpés de cette manière !]
	\begin{figure}
		\includegraphics{resources/q39_horizontal}
	\end{figure}
\end{problem}

\begin{problem}{40.}
	Les quatre sommets d’un parallélogramme ont été choisis sur les nœuds d’une
	feuille quadrillée. Il s’avère qu’aucun des côtés ni l’intérieur du
	parallélogramme ne contient d’autres nœuds de la feuille. Prouvez que l’aire
	d’un tel parallélogramme est égale à celle d’un carré de la feuille.
	\begin{figure}
		\includegraphics{resources/taskbook-24}
	\end{figure}
\end{problem}

\begin{problem}{41.}
	Dans les conditions de la question 40, $a$ nœuds se trouvent à l’intérieur
	et $b$ sur les côtés du parallélogramme. Trouvez son aire.
\end{problem}

\begin{problem}{42.}
	L’énoncé analogue à la question 40 est-il vrai pour les parallélépipèdes
	dans l’espace à 3 dimensions ?
\end{problem}

\begin{problem}{43.}
	Les nombres de Fibonacci (ou nombres du lapin) forment une suite $1, 2, 3,
	5, 8, 13, 21, 34, \dotsc$ où $a_{n+2}=a_{n+1}+a_n$ pour tout $n=1, 2,
	\dotsc$ ($a_n$ est le $n$-ième nombre de la suite). Trouvez le plus grand
	commun diviseur des nombres $a_{100}$ et $a_{99}$.
\end{problem}

\begin{problem}{44.}
	Trouvez le nombre (de Catalan) de façons de découper un $n$-gone convexe en
	triangles en traçant ses diagonales non-intersectantes. Par exemple,
	$c(4)=2$, $c(5)=5$, $c(6)=14$. Comment peut-on calculer $c(10)$ ?
	\begin{figure}
		\includegraphics{resources/taskbook-281}
		\qquad
		\includegraphics{resources/taskbook-282}
	\end{figure}
\end{problem}

\begin{problem}{45.}
	Lors d'une coupe avec $n$ équipes participantes, chaque équipe perdante est
	éliminée, et le gagnant final est décidé après $n-1$ matchs. Le programme du
	tournoi peut être représenté symboliquement, par exemple $((a,(b,c)),d)$
	signifie que $b$ joue contre $c$, le gagnant rencontre $a$, et le gagnant de
	ces deux rencontre $d$. Quel est le nombre de programmes différents pour 10
	équipes ?
	\begin{itemize}
		\item Pour 2 équipes, nous avons seulement $(a,b)$, et le nombre est 1.
		\item Pour 3 équipes, il y a seulement $((a,b),c)$, ou $((a,c),b)$, ou
		$((b,c),a)$, et le nombre est 3.
		\item Pour 4 équipes :
			\begin{equation*}
				\begin{array}{@{}cccc@{}}
					(((a,b),c),d) & \quad\;(((a,c),b),d) & \quad\;(((a,d),b),c) & \quad\;(((b,c),a),d) \\
					(((b,d),a),c) & \quad\;(((c,d),a),b) & \quad\;(((a,b),d),c) & \quad\;(((a,c),d),b) \\
					(((a,d),c),b) & \quad\;(((b,c),d),a) & \quad\;(((b,d),c),a) & \quad\;(((c,d),b),a) \\
					((a,b),(c,d)) & \quad\;((a,c),(b,d)) & \quad\;((a,d),(b,c))
				\end{array}
			\end{equation*}
	\end{itemize}
\end{problem}

\begin{problem}{46.}
    Joindre \( n \) points \( 1, 2, \dotsc, n \) par des intervalles (\( n-1 \) d'entre eux) pour obtenir un arbre.
    Combien d'arbres différents peut-on obtenir (le cas \( n=5 \) est déjà intéressant !) ?

    \( n=2 \) : \quad \includegraphics{resources/taskbook-291}\,,\quad le nombre est 1 ;

    \( n=3 \) : \quad
    \includegraphics{resources/taskbook-292}\,,\quad
    \includegraphics{resources/taskbook-293}\,,\quad
    \includegraphics{resources/taskbook-294}\,,\quad
    le nombre est 3 ;

    \( n=4 \) : \quad\def\quad{\hskip.7em}
    \(\vcenter{\hbox{\includegraphics{resources/taskbook-295}}}\),\quad
    \(\vcenter{\hbox{\includegraphics{resources/taskbook-296}}}\),\quad
    \(\vcenter{\hbox{\includegraphics{resources/taskbook-297}}}\),\quad
    \(\vcenter{\hbox{\includegraphics{resources/taskbook-298}}}\),\quad
    \(\vcenter{\hbox{\includegraphics{resources/taskbook-299}}\hbox{\includegraphics{resources/taskbook-290}}
    \vskip-8pt
    \hbox to50bp{\dotfill}}\),\\
    \null\hspace{\parindent}\phantom{\( n=4 \):}\quad le nombre est 16.
\end{problem}

\begin{problem}{47.}
    Une permutation \( (x_1, x_2, \dotsc, x_n) \) des nombres \( \{1, 2, \dotsc, n\} \) est appelée un
    \emph{serpent} (de longueur \( n \)) si \( x_1 < x_2 > x_3 < x_4 \dotsb \).

    \begin{note}{Exemple :}
        \begin{equation*}
            \begin{aligned}[t]
                &\begin{aligned}[t] n=2, \text{\ \ uniquement\ \ } 1<2, \end{aligned} &&\text{le nombre est }1, \\
                &\hskip-\nulldelimiterspace\mathord{\left.\begin{aligned} n=3, \hphantom{\text{\ \ uniquement\ \ }} 1&<3>2 \\
                2&<3>1\end{aligned} \right\}}, && \text{le nombre est }2, \\
                &\hskip-\nulldelimiterspace\mathord{\left.\begin{aligned} n=4, \hphantom{\text{\ \ uniquement\ \ }} 1&<3>2<4 \\
                1&<4>2<3 \\
                2&<3>1<4 \\
                2&<4>1<3 \\
                3&<4>1<2\end{aligned} \right\}},
                &&\text{le nombre est }5. \\
            \end{aligned}
        \end{equation*}
    \end{note}
    Trouvez le nombre de serpents de longueur \( 10 \).
\end{problem}

\begin{problem}{48.}
    Soit \( s_n \) le nombre de serpents de longueur \( n \) :
    \begin{equation*}
        s_1=1, \quad s_2=1, \quad s_3=2, \quad s_4=5, \quad s_5=16, \quad s_6=61.
    \end{equation*}
    Montrez que la série de Taylor de la tangente est
    \begin{equation*}
        \tan x=1\, \frac{x^1}{1!}+2\, \frac{x^3}{3!}+16\, \frac{x^5}{5!}+\dots=
        \textstyle\sum\limits_{k=1}^{\infty} s_{2k-1}\, \frac{x^{2k-1}}{(2k-1)!}.
    \end{equation*}
\end{problem}

\begin{problem}{49.}
    Trouvez la somme de la série
    \begin{equation*}
        1+1\, \frac{x^2}{2!}+5\, \frac{x^4}{4!}+61\, \frac{x^6}{6!}+\dots=
        \textstyle\sum\limits_{k=0}^{\infty} s_{2k}\,\frac{x^{2k}}{(2k)!}.
    \end{equation*}
\end{problem}

\begin{problem}{50.}
    Pour \( s>1 \), démontrez l'identité :
    \begin{equation*}
        \textstyle\prod\limits_{p=2}^{\infty} \frac{1}{1-\frac{1}{p^s}}=\textstyle\sum\limits_{n=1}^{\infty} \frac{1}{n^s}.
    \end{equation*}
    (Le produit porte sur tous les nombres premiers \( p \), et la somme sur tous les nombres naturels \( n \).)
\end{problem}

\begin{problem}{51.}
    Trouvez la somme de la série :
    \begin{equation*}
        1+ \frac{1}{4}+ \frac{1}{9}+\dots=\textstyle\sum\limits_{n=1}^{\infty} \frac{1}{n^2}.
    \end{equation*}
    [Montrez qu'elle est égale à \(\nicefrac{\pi^2}{6}\), c'est-à-dire environ \(\nicefrac{3}{2}\).]
\end{problem}

\begin{problem}{52.}
    Trouvez la probabilité d'irréductibilité d'une fraction \(\nicefrac{p}{q}\) (cela est défini comme suit :
    dans le disque \( p^2+q^2 \leqslant R^2 \), nous comptons le nombre \( N \) de vecteurs avec \( p \) et \( q \) entiers
    n'ayant pas de diviseur commun supérieur à 1, après quoi la probabilité d'irréductibilité est la
    limite du rapport \(\nicefrac{N(R)}{M(R)}\), où \( M(R) \) est le nombre de points entiers dans le disque (\( M \sim \pi R^2 \)).)
    \begin{figure}
        \includegraphics{resources/taskbook-36}\\
        \footnotesize \( M(5)=81 \), \( N(5)=44 \), \(\nicefrac{N}{M} = \nicefrac{44}{81} \)
    \end{figure}
\end{problem}

\begin{problem}{53.}
    Pour la suite des nombres de Fibonacci \( a_n \) du problème 43, trouvez la limite du rapport
    \( a_{n+1}/a_n \) lorsque \( n \) tend vers l'infini :
    \begin{equation*}
        \frac{a_{n+1}}{a_n}=2,\ \frac 32,\ \frac53, \ \frac85, \ \frac{13}8,
        \ \frac{34}{21}.
    \end{equation*}
    [La réponse est \enquote{le nombre d'or},
    \(\frac{\sqrt{5}+1}{2\vphantom)} \approx 1.618\). C'est le rapport des côtés d'une carte qui reste
    similaire à elle-même après avoir retiré le carré dont le côté est le plus petit côté de la carte,
    \(\frac{AB}{BC}=\frac{PC}{CD}\). Comment le nombre d'or est-il lié à un pentagone régulier et à une étoile à cinq branches ?]
    \begin{figure}
        \includegraphics{resources/taskbook-37}
    \end{figure}
\end{problem}

\begin{problem}{54.}
    Calculez la fraction continue infinie
    \begin{equation*}
        1+\cfrac{1}{2+\cfrac{1}{1+\cfrac{1}{2+\cfrac{1}{1+\cfrac{1}{2+\ldots}}}}}=
        a_0+\cfrac{1}{a_1+\cfrac{1}{a_2+\cfrac{1}{a_3+\dots}}}
    \end{equation*}
    avec \( a_{2k}=1 \) et \( a_{2k+1}=2 \) (c'est-à-dire, trouvez la limite des fractions
    \begin{equation*}
        a_0+\cfrac{1}{a_1+\cfrac{1}{a_2+{\atop{\ddots \atop {}} + \cfrac{1}{a_n}}}}
    \end{equation*}
    pour \( n \to \infty \)).
\end{problem}

\begin{problem}{55.}
    Trouvez les polynômes
    \begin{equation*}
        y=\cos 3 (\arccos x),\ y=\cos 4 (\arccos x),\
        y=\cos n (\arccos x),
    \end{equation*}
    où \( |x| \leqslant 1 \).
\end{problem}

\begin{problem}{56.}
    Calculez la somme des puissances \( k \)-ièmes des \( n \) racines \( n \)-ièmes complexes de l'unité.
\end{problem}

\begin{problem}{57.}
    Dans le plan $(x, y)$, tracez les courbes définies paramétriquement par :
    \begin{equation*}
        \{x=\cos 2t, y=\sin 3t\},\quad
        \{x=t^3-3t, y=t^4-2t^2\}.
    \end{equation*}
    \vspace{-2\baselineskip}%supprimez cet espace vertical si votre traduction a du texte après l'équation
\end{problem}

\begin{problem}{58.}
    Calculez $\int_0^{2\pi} \sin^{100} x\,dx$ (avec une erreur ne dépassant pas 10 % de la réponse).
\end{problem}

\begin{problem}{59.}
    Calculez $\int_1^{10} x^x\,dx$ (avec une erreur ne dépassant pas 10 % de la réponse).
\end{problem}

\begin{problem}{60.}
    Trouvez l'aire d'un triangle ayant des angles $(\alpha, \beta, \gamma)$ sur une sphère de rayon 1,
    dont les côtés sont des grands cercles (sections de la sphère par des plans passant par son centre).

    \begin{note}{Réponse :}
        $S=\alpha+\beta+\gamma-\pi$ (par exemple, pour un triangle avec
        trois angles droits, $S=\nicefrac{\pi}{2}$, soit un huitième de la surface totale de la sphère).
        \begin{figure}
            \null\hfill
            \includegraphics{resources/taskbook-44}
            \hfill
            \includegraphics{resources/taskbook-45}
            \hfill\null
        \end{figure}
    \end{note}
\end{problem}

\begin{problem}{61.}
    Un cercle de rayon $r$ roule (sans glisser) à l'intérieur d’un cercle de rayon 1.
    Tracez la trajectoire complète d’un point du cercle roulant (cette trajectoire est appelée hypocycloïde)
    pour $r=\nicefrac{1}{3}$, pour $r=\nicefrac{1}{4}$, pour $r=\nicefrac{1}{n}$, et pour $r=\nicefrac{1}{2}$.
\end{problem}

\begin{problem}{62.}
    Dans une classe de $n$ élèves, estimez la probabilité qu’il y ait deux élèves ayant la même date de naissance. Est-elle élevée ou faible ?

    \begin{note}{Réponse :}
        (très) élevée si le nombre d’élèves est (bien) supérieur à $n_0$,
        (très) faible s’il est (bien) inférieur à $n_0$, et trouvez ce qu’est réellement ce $n_0$
        (quand la probabilité $p \approx \nicefrac{1}{2}$).
    \end{note}
\end{problem}

\begin{problem}{63.}
    La loi de Snell (ou de Snellius) stipule que l’angle $\alpha$ formé par un rayon de lumière avec la normale aux couches d’un milieu stratifié satisfait l’équation :
    \begin{equation*}
        n(y) \sin \alpha=\text{const},
    \end{equation*}
    où $n(y)$ est l’indice de réfraction de la couche à la hauteur $y$ (la quantité $n$ est inversement proportionnelle à la vitesse de la lumière dans le milieu, en prenant sa vitesse dans le vide comme 1 ; dans l’eau, $n=\nicefrac{4}{3}$).
    \begin{figure}
        \null\hfill
        \includegraphics{resources/taskbook-47}
        \hfill
        \includegraphics{resources/taskbook-471}
        \hfill\null
    \end{figure}

    Tracez les trajectoires des rayons dans le milieu \enquote{air au-dessus d’un désert}, où l’indice $n(y)$ atteint un maximum
    à une certaine hauteur.
    (Une solution à ce problème explique les mirages dans un désert pour ceux qui comprennent comment les trajectoires des rayons émanant d’objets sont liées aux images.)
\end{problem}

\begin{problem}{64.}
    Inscrivez dans un triangle aigu $ABC$ un triangle $KLM$ de périmètre minimal
    (avec son sommet $K$ sur $AB$, $L$ sur $BC$, $M$ sur $CA$).
    \begin{figure}
        \includegraphics{resources/taskbook-48}
    \end{figure}

    \begin{note}{Indice :}
        La réponse pour les triangles non aigus n’est pas similaire à la belle réponse pour les triangles aigus.
    \end{note}
\end{problem}

\begin{problem}{65.}
    Calculez la valeur moyenne de la fonction $\nicefrac{1}{r}$ (où
    $r^2=x^2+y^2+z^2$, $r$ est la distance à l’origine) sur une sphère de rayon
    $R$ centrée au point $(X,Y,Z)$.

    \begin{note}{Indice :}
        Le problème est lié à la loi de la gravitation de Newton et à la loi de Coulomb en théorie électrique.
        Dans la version bidimensionnelle du problème, la fonction doit être remplacée par $\ln r$, et la sphère par le cercle.
    \end{note}
\end{problem}

\begin{problem}{66.}
    Le fait que $2^{10}=1024 \approx 10^3$ implique
    $\log_{10} 2 \approx 0,3$. Estimez à quel point ils diffèrent, et calculez $\log_{10} 2$ à trois décimales.
\end{problem}

\begin{problem}{67.}
    Trouvez $\log_{10} 4$, $\log_{10} 8$,
    $\log_{10} 5$, $\log_{10} 50$, $\log_{10} 32$, $\log_{10} 128$,
    $\log_{10} 125$, $\log_{10} 64$ avec la même précision.
\end{problem}
\begin{problem}{68.}
    En utilisant $7^2 \approx 50$, trouvez une valeur approximative de $\log_{10} 7$.
\end{problem}

\begin{problem}{69.}
    En connaissant $\log_{10} 64$ et $\log_{10} 7$, trouvez $\log_{10} 9$, $\log_{10} 3$,
    $\log_{10} 27$, $\log_{10} 6$, $\log_{10} 12$.
\end{problem}

\begin{problem}{70.}
    En utilisant $\ln (1+x) \approx x$ ($\ln$ est $\log_e$), trouvez $\log_{10} e$ et
    $\ln 10$ à partir de la relation\footnote{Le nombre d’Euler $e = 2{,}71828\dots$ est défini comme la limite de la suite
    $\left(1+\frac{1}{n}\right)^n$ lorsque $n\to \infty$, et est égal à la somme de la série
    $1+\frac{1}{1!} +\frac{1}{2!}+\frac{1}{3!}+\dotsb$. Il peut aussi être défini via la formule citée pour
    $\ln (1+x)$ : $\lim\limits_{x\to 0}\frac{\ln(1+x)}{x} = 1$.}
    %
    \begin{equation*}
        \log_{10} a=\frac{\ln a}{\ln 10}
    \end{equation*}
    ainsi qu’à partir des valeurs de $\log_{10} a$ calculées précédemment (par exemple, pour $a=128/125, 1024/1000$,
    et ainsi de suite).

    [Les solutions des problèmes 65--69 permettent après une demi-heure de construire une table de logarithmes
    à quatre chiffres de n’importe quels nombres en utilisant les produits des nombres déjà trouvés comme données de base et la formule
    \begin{equation*}
        \ln (1+x) \approx x-\frac{x^2}{2}+\frac{x^3}{3}-\frac{x^4}{4}+\dotsb,
    \end{equation*}
    pour les corrections.] (C’est ainsi que Newton a compilé une table
    de logarithmes à 40 chiffres !)
\end{problem}

\begin{problem}{71.}
    Considérez la suite des puissances de deux : $1$, $2$, $4$, $8$, $16$, $32$, $64$,
    $128$, $256$, $512$, $1024$, $2048, \dotsc$ Parmi les douze premiers nombres, quatre ont une expression décimale
    commençant par 1, et aucun ne commence par 7.

    Montrez que dans la limite $n \to \infty$, le premier chiffre des nombres $2^m$,
    $0\leqslant m \leqslant n$, apparaît avec une certaine fréquence :
    $p_1 \approx 30\%, p_2 \approx 18\%, \dotsc, p_9 \approx 4\%$.
\end{problem}

\begin{problem}{72.}
    Vérifiez le comportement des premiers chiffres des puissances de trois : $1,
    3, 9, 2, 8, 2, 7, \dotsc$ Montrez que, dans la limite, on obtient également
    certaines fréquences et, de plus, les mêmes que pour les puissances de deux.
    Trouvez une formule exacte pour $p_1, \dotsc, p_9$.

    \begin{note}{Indice :}
        Le premier chiffre d’un nombre $x$ est déterminé par la partie fractionnaire
        du nombre $\log_{10} x$. Par conséquent, il faut considérer la suite des parties fractionnaires des
        nombres $m \alpha$, où $\alpha=\log_{10} 2$.
    \end{note}
    Montrez que ces parties fractionnaires sont distribuées uniformément sur l’intervalle de 0 à 1 :
    parmi les $n$ parties fractionnaires des nombres $m \alpha$, $0 \leqslant m<n$,
    un sous-intervalle $A$ contiendra la quantité~$k_n (A)$ telle que, pour $n \to \infty$,
    $\lim (k_n (A)/n)=(\text{la longueur du sous-intervalle~$A$})$.
\end{problem}

\begin{problem}{73.}
    Soit $g\colon M \to M$ une application lisse d’un domaine borné $M$ sur lui-même, qui est
    bijective et préserve les aires (ou volumes dans le cas multidimensionnel) des domaines.

    Montrez que dans tout voisinage $U$ de n’importe quel point de $M$ et pour tout $N$, il existe un point $x$
    tel que $g^T x$ soit également dans $U$ pour un certain entier $T>N$ (\enquote{théorème de récurrence}).
\end{problem}

\begin{problem}{74.}
    Soit $M$ la surface d’un tore (avec les coordonnées $\alpha \pmod{2\pi}$, $\beta \pmod{2\pi}$),
    et
    \begin{equation*}
        g(\alpha, \beta)=(\alpha+1, \beta+ \sqrt{2}) \pmod{2\pi}.
    \end{equation*}
    Montrez que la suite des points
    $\{g^T (x)\}$, $T=1, 2, \dotsc$, est dense partout dans le tore.
\end{problem}

\begin{problem}{75.}
    Dans les notations du problème 74, soit
    \begin{equation*}
        f(\alpha, \beta)=(2\alpha+\beta,\alpha+\beta) \pmod{2\pi}.
    \end{equation*}
    Montrez qu’il existe un sous-ensemble dense partout du tore composé de points périodiques $x$ (c’est-à-dire tels que
    $f^{T (x)} x=x$ pour un certain entier $T>0$).
\end{problem}

\begin{problem}{76.}
    Dans les notations du problème 74, montrez que, pour presque tous les points $x$ du tore,
    la suite des points $\{g^T (x)\}$, $T=1, 2, \dotsc$, est dense partout dans le tore
    (les points $x$ sans cette propriété constituent un ensemble de mesure nulle).
\end{problem}

\begin{problem}{77.}
    Dans les problèmes 74 et 76, montrez que la suite $\{g^T (x)\}$, $T=1, 2, \dotsc$, est répartie
    uniformément sur le tore : si un domaine $A$ contient $k_n(A)$ points parmi les $n$ avec $T=1, 2, \dotsc,n$, alors
    \begin{equation*}
        \lim_{n \to \infty} \frac{k_n(A)}{n}=\frac{\operatorname{mes} A}{\operatorname{mes} M}
    \end{equation*}
    (par exemple, pour un domaine mesurable au sens de Jordan $A$ de mesure $\operatorname{mes} A$).
\end{problem}
\clearpage

\vfill
\begin{note}{Note pour le problème 13.}
    J’ai essayé d’illustrer avec ce problème la différence entre les approches des tâches par les mathématiciens et les physiciens, dans mon article invité dans le journal \enquote{Physics -- Uspekhi} pour le jubilé de Noël 2000. Mon succès a dépassé de loin ce que je voulais : les éditeurs, contrairement aux enfants d’âge préscolaire, sur lesquels je me basais pour mon expérience, n’ont pas réussi à résoudre le problème, et l’ont donc modifié pour correspondre à ma réponse de \SI{4}{\mm} comme suit : au lieu de \enquote{de la première page du volume 1 à la dernière page du volume 2}, ils ont imprimé \enquote{de la \emph{dernière} page du volume 1 à la \emph{première} page du volume 2}.

    Cette histoire vraie est si improbable que je l’inclus ici : la preuve est la version des éditeurs publiée par le journal.
\end{note}
\clearpage
\null\vfill
\noindent
Traduction Anglais - Français :\\
\null\quad Romain Tavenard et Jonathan Demeyer\\
\\
Conception et mise en page :\\
\null\quad Konrad Renner et Christian Stussak\\
\\
\\
De l'édition russe :\\
\null\quad \textrussian{В. И. Арнольд: Задачи для детей от 5 до 15 лет}\\
\null\quad Moscou, MCCME, 2004\\
\null\quad ISBN 5-94057-183-2\\
\\
\\
Crédits photo page de titre :\\
\null\quad Archives de l’institut de recherches mathématiques d'Oberwolfach\\
\\
Version :\\
\null\quad \today\\
\\
Ce livre est disponible sous licence CC BY-NC-SA 3.0 sur la plateforme IMAG\-I\-NARY : \href{http://www.imaginary.org/background-materials}{www.imaginary.org/background-materials}.\\
IMAGINARY est un projet de l’institut de recherches mathématiques d'Oberwolfach soutenu par la Fondation Klaus-Tschira.\end{document}
